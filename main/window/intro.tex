The Nonlinear Fourier Transform (NFT) stands as a potent tool for the analysis and comprehension of nonlinear systems, especially those of a periodic or quasi-periodic nature~\cite{ZakharovShabat1972, Ablowitz1981, Kamalian2018}. In recent years, its application has extended across various domains, including nonlinear and quantum optics~\cite{Gelash2019PRL, Mullyadzhanov2019, Chekhovskoy2019_PRL}.

A significant advantage of NFT lies in its capability to compute signal propagation over extended distances through a three-step process~\cite{Yousefi2014I}: direct NFT, nonlinear spectrum evolution, and inverse NFT. This approach enables the precise modeling of intricate interactions between chromatic dispersion and nonlinear phenomena, such as the Kerr effect, which manifest when optical signals traverse substantial distances. Furthermore, NFT excels at handling signals with arbitrary boundary conditions, rendering it an ideal choice for continuous signal processing.

In light of this, the interest in numerical techniques for solving direct and inverse problems has grown~\cite{Boffetta1992a, Belai2006, Frumin2015_TIB, 10026447}. Comprehensive reviews of existing methods are available in~\cite{Yousefi2014II, Turitsyn2017Optica, Vasylchenkova2019a}. Notably, there exist fast methods for direct and inverse NFT calculations, characterized by lower asymptotic complexity compared to conventional approaches. These are referred to as the fast nonlinear Fourier transform (FNFT)~\cite{Wahls2013, Wahls2015, Wahls2015a, Wahls2016, Vaibhav2018, Wahls2018, Chimmalgi2019}.

Concerning nonlinear interactions, NFT emerges as a pivotal tool for enhancing the performance of optical communication systems and augmenting throughput~\cite{Kamalian2018, Turitsyn2017Optica, Le2014, Prilepsky2014, essiambre2012capacity, Civelli2019, Sedov:18}. Moreover, NFT yields a more precise depiction of the signal and its nonlinear distortions, fostering the development of more efficient communication systems. This encompasses improved modulation and detection schemes, along with advanced error correction techniques~\cite{Frumin2017, Gui2017a, Aref2018_JLT}.

Nonetheless, challenges persist in employing NFT for optical telecommunications. The computational complexity of NFT algorithms can be substantial, especially for systems featuring numerous channels or a broad spectrum of signal parameters. Additionally, implementing NFT-based signal processing may necessitate specialized hardware or software, introducing complexity and elevating system costs.

Notwithstanding these challenges, the potential advantages of NFT in optical telecommunications render it an enticing area of research, with the potential to revolutionize the processing and transmission of optical signals within future communication networks.

The application of NFT to continuous signals presents a notable challenge, as merely segmenting the signal into non-overlapping portions using a sliding window technique falls short of efficient processing.

This article introduces a novel method for enhancing the accuracy and efficiency of continuous signal processing with vanishing boundary conditions using NFT. This method aims to address the challenges associated with traditional NFT algorithms and offers a promising solution for signal processing across diverse fields. Through an in-depth exploration of the underlying mathematical principles and numerical simulations, we demonstrate the effectiveness of this new approach across various scenarios.

The process of partitioning a continuous signal into non-overlapping segments, commonly referred to as the "sliding window" technique, serves as a fundamental method in signal processing. This technique seeks to extract meaningful insights from a signal by independently analyzing each segment or window.

In the realm of nonlinear signal processing, the sliding window technique can be harnessed to transform a continuous signal into a discrete representation. This transformation is achieved by applying a nonlinear operation, such as NFT, to each windowed segment of the signal. The resulting discrete representation serves as a basis for extracting valuable information about the signal, encompassing dominant frequency components and nonlinear characteristics.

Furthermore, the size of the window, denoting the number of samples within each windowed segment, emerges as a critical parameter influencing the accuracy and resolution of the analysis. A larger window size enhances resolution but elevates computational complexity. Conversely, a smaller window size reduces computational demands but sacrifices resolution.

We propose an enhanced method that combines chromatic dispersion compensation (CDC) with the sliding window technique for continuous signal processing within optical communication systems. By pre-compensating chromatic dispersion, we mitigate overlap and distortion among adjacent characters, ultimately improving the performance and applicability of NFT. This approach bears similarities to reducing nonlinearity using neural networks~\cite{9324921, 8535144, freire2021performance, zibar2015machine, cartledge2017digital}. Our method enhances both accuracy and efficiency while resolving issues associated with continuous signal processing, resulting in heightened system performance and increased throughput, particularly in optical telecommunications.