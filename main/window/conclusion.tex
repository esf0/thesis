In summary, this article introduces a novel method for enhancing the accuracy and efficiency of continuous signal processing using the Nonlinear Fourier Transform (NFT), a complex problem in the field of signal processing. The proposed method effectively addresses the limitations of traditional NFT algorithms and offers a promising solution for signal processing applications across diverse fields.

We have conducted an extensive analysis of the underlying mathematical principles and supported our findings with numerical simulations, illustrating the efficacy of this proposed approach in various scenarios. Our methodology is applied to both the Schr\"odinger equation, simulating single polarization signal propagation in optical systems, and the Manakov equation, which characterizes light propagation in a two-polarization optical fiber.

The method leverages a sliding window technique, dividing a continuous signal into non-overlapping segments, with each segment undergoing a nonlinear operation like NFT to transform it into a discrete representation. The size of the window, a crucial parameter influencing the accuracy and resolution of the analysis, is optimized through numerical simulations. There exists a direct correlation between the window size employed in a processing system and the overall system speed. Larger window sizes allow for more data to be processed at once, resulting in increased processing speed. However, it's important to note that there may be diminishing returns as window sizes become excessively large, as the heightened processing power required to handle larger datasets might offset the gains in processing speed. Therefore, achieving an optimal balance between window size and processing power is imperative for system performance optimization.

We evaluated the proposed method using a single-channel wavelength division multiplexing (WDM) signal, a common feature in optical communication systems. The results demonstrate that the proposed method yields a lower bit error rate (BER) compared to the traditional NFT algorithm, underscoring its effectiveness.

This approach paves the way for practical implementation in real communication systems. Its adaptability and flexibility make it suitable for a wide array of signal processing applications spanning various fields, including nonlinear optics and quantum optics.

Future research in this domain could delve into investigating the impact of point drift on signal reconstruction quality using NFT, as well as the development of optimization techniques for various parameters within the proposed method, such as window size, with the aim of further enhancing accuracy and efficiency.

As a final note, we want to stress that an extensive part of this analysis was related to code creation, which can be found in the GitHub repository at \href{https://github.com/esf0/nft-processing}{nft-processing}. The repository includes comprehensive code examples and Jupyter notebooks to reproduce the results.