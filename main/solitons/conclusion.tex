\subsection{Conclusion}

This study demonstrates that solitons are an integral component of optical communication signals. From the analysis of single symbols, which reveals the close relationship between solitons and the signal's internal structure—such as independent subcarriers in OFDM or different channels in \gls{wdm}—to the examination of complex \gls{wdm} signals that may contain dozens or even hundreds of solitons, it is evident that nonlinear structures can have a significant influence on signal formation. This realization provides us with valuable insights that could inform further research and enhancements in modulation techniques or signal formats.

We observed that the statistical behavior of \gls{wdm} signals offers clues about the internal thermodynamics of the `soliton gas' present within these signals. Leveraging such statistics could lead to a more profound understanding of the nonlinear interactions occurring within the signal, ultimately contributing to improvements in transmission quality. If we can decipher the effects, we may be able to devise methods to mitigate or even exploit them to our advantage, similar to the use of solitons for data transmission. While the subsequent stages of this research are outside the context of this thesis, it is clear that the nonlinear Fourier properties of signals open up numerous possibilities for future exploration.
