Since linear, nonlinear, and noise effects manifest themselves simultaneously when we transmit data over fibre-optic communication lines, such systems are well fit to be dealt with using the latest advances in machine learning methods. Using these methods, it is possible to solve the problem of multidimensional optimisation (for example, in terms of data transmission quality and data throughput maximisation) without having to iterate through all possible parameter values.
Of particular relevance is the problem of identifying
some internal features and patterns of the transmitted data, where neural networks can be used to simulate various
effects that affect the signal when it propagates through a noisy nonlinear medium. In other words, neural networks can be used to simulate nonlinear transformations without the need for direct calculation of these transformations. The advantage lies in the speed and versatility of the transformation, as well as the flexibility and adaptability of operations based on a neural network: the network does not know what data it processes; it looks for the necessary features in the data that affect the final result, and then extracts them. This process is called feature extraction. Thus, if we want to calculate a certain value of a function, instead of (possibly) complex calculations, we can use a pre-trained network that, with a pre-known number of operations, will give the desired result. The difficulty is that the neural networks need to be trained up-front on the known data.
Another advantage of signal processing based on neural networks is that networks can reduce the noise component present in the analysed data [38]. In practice, we almost always encounter a situation where there is some noise in the data, for example, due to the finite accuracy of measurements, and its presence may be critical for accurate data processing methods. A neural network can effectively filter out unnecessary information within itself, leaving only the basic features needed for a specific task. Note that one of the disadvantages of using neural networks is the final accuracy of the result attainable with the use of a trained network. However, in practice, the accuracy of a neural network is sufficient for most tasks and sometimes even exceeds the accuracy of existing numerical methods if the necessary set of training data is available.

The first direction of utilising \acrshort{nn}s for \acrshort{nfdm} systems consists in applying the additional \acrshort{nn}-based processing unit at the receiver to compensate the emerging line impairments and deviations from the ideal model \cite{gdd18,kwp19,kkp19,kpk20,kkp21}. But, despite ensuing transmission quality improvement, this type of \acrshort{nn} usage brings about the additional complexity of the receiver. In the other approach, the \acrshort{nft} operation at the receiver is entirely replaced by the \acrshort{nn} element. It has been shown that this approach, indeed, results in a considerable improvement of the \acrshort{nft}-based transmission system functioning \cite{ymm19,jgy18,wxz20}. But, despite the benefits rendered by such a \acrshort{nn} utilisation, the \acrshort{nn}s emulating the \acrshort{nft} operation have so far been mostly used in the \acrshort{nfdm} systems operating with solitons only, and the \acrshort{nn} structure used there was relatively simple.  In the only work related to the continuous NF spectrum recovery~\cite{zhang2021direct}, a standard ``imageInputLayer'' \acrshort{nn} (developed originally for hand-written digits recognition) from MATLAB 2019a deep learning toolbox was adapted to process the signals of a special form. Such an approach, evidently, has limited applicability and flexibility and is not optimal neither in terms of the result's quality nor in the complexity of signal processing. 
In our current work, we demonstrate how this direction can be significantly extended and optimised by using \acrfull{nn} for modeling the continuous part of the \acrfull{nf} spectrum. We show how \acrshort{nn}s could be used to detect the discrete spectrum and demonstrate the use of optimization tools to choose the best \acrshort{nn} design. Our research aims to establish a foundation for creating highly effective \acrshort{nfdm} systems that work well with any channel. Additionally, we don't just focus on finding the NF spectrum \( r(\xi) \), but also on the \( b \)-coefficient, which can be used in the most effective method of \acrshort{nfdm} known as \( b \)-modulation.


The foundation of this chapter's content is the outcomes reported in~\cite{sedov2020application, sedov2021neural, sedov2021direct}. This chapter is structured as follows: At the close of the introduction, we detail the data format employed throughout the chapter. Subsequent to this, we start with an initial exploration of utilizing \acrfull{nn} to execute both forward and inverse \acrfull{nft} operations on signals without solitons, which have been omitted from the dataset. The following section is dedicated to signals that do contain solitons, focusing on the application of \acrshort{nn} to estimate soliton quantities. The core section presents a case study in which a neural network is applied to predict the continuous NF spectrum of soliton-free signals in case of noise interference. The chapter concludes with a discussion on prospective routes for future research.


\subsection{Data format}


% Start of LaTeX code
In the following examination of optical signals, we utilize the \gls{wdm} format (see Eq.~(\ref{eq:wdm_nlse})). In this section, we express the \gls{wdm} signal in dimensionless units, a practical necessity for applying the \Gls{nft} in our analysis. The mathematical representation of a \acrfull{rz} \acrshort{wdm} symbol is given by:

\begin{equation}
s(t) = \frac{1}{Q} \sum_{k=1}^{M} C_k e^{i \omega_k t} f(t), \quad 0 \leq t < T,
\label{eq:wdm}
\end{equation}
where $M$ is a number of \acrshort{wdm} channels, $\omega_k$ is a carrier frequency of the $k$-th channel, $C_k$ corresponds to the digital data in $k$-th channel, and $T$ defines the symbol interval.
$Q$ in (\ref{eq:wdm}) is the normalisation factor that we use to set the required energy for each signal (the total signal energy is calculated according to Eq.~(\ref{eq:snr})).
In Eq.~(\ref{eq:wdm}), each coefficient \( C_k \) represents a complex number selected from a constellation diagram of a given size, meaning it is randomly chosen from a predetermined set of possible values with uniform probability.
For our NF decomposition analysis each time we use a single signal of the form given in Eq.~(\ref{eq:wdm}). 
\( f(t) \) is the waveform of the carrier pulse, defined as (and represented on Fig.~\ref{fig:f_shapes}):
\begin{equation}
f(t) = 
\begin{cases} 
\frac{1}{2} \left[ 1 - \cos \left( \frac{4\pi t}{T} \right) \right], & 0 \leq t \leq \frac{T}{4} \text{ or } \frac{3T}{4} \leq t \leq T, \\
1, & \frac{T}{4} < t < \frac{3T}{4}.
\end{cases}
\label{eq:wdm_envelope}
\end{equation}


