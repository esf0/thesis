% Conclusion in LaTeX
In summary, machine learning (ML) and neural networks (NN) are cutting-edge technologies that have garnered substantial interest in the realm of nonlinear signal processing and optical communications. The neural network architecture proposed in this study illustrates the initial yet promising capability of NNs for the intricate task of analyzing and (de)modulating the complex optical signals prevalent in communication systems. Such advances herald the potential to enhance current optical communication infrastructures by leveraging intelligent algorithms, such as digital back-propagation based on Nonlinear Fourier Transform (NFT) and NN, even without comprehensive knowledge of the nonlinear phenomena that influence signal quality.
Our research indicates that NNs are not only capable of dissecting the internal structure of optical signals but also of forging new signal patterns through the use of autoencoders. The application of NNs to NFT opens new research frontiers in understanding the inherently nonlinear properties of signal structures and their evolution. 


Our goal in this work was to demonstrate that the NN can be successfully used for performing the NFT operation, in particular, for computing the profile of continuous NF spectrum. 
Note that our interest was not only the computation of the continuous NF spectrum, i.e. the nonlinear transformation, but the possibility to denoise signal using NNs. 

Once again, we emphasize that Bayesian optimization does not always give the "best" set of parameters.
It provides a subspace of hyperparameters in which neural networks with such parameters are best trained on the available dataset.
Due to the fact that neural networks are universal approximators, any sufficiently complex architecture can be trained for a specific task.
We can expect that the optimization process can converge endlessly towards increasing the complexity of the network.
However, this is not suitable for our task, where we want to minimize the complexity of the network while improving the accuracy of the work. 
Therefore, we simultaneously limited the number of trainable parameters in the NN during optimization.
In our case, during the optimization process, we found an architecture that gives us the best metric value~(\ref{eq:metric}) and we chose it as the desired architecture. Further, the optimization process could converge to another subspace of hyperparameters, but we stick to the point with the minimum value of the loss function.

We found that this NN, indeed, can perform the NFT operation and denoise the received NF spectrum: the denoising effect is pronounced at medium to high noise levels. 
To achieve this effect, several realisations of the noise are needed for the neural network to "understand" the influence of noise on the signal. 
As expected, denoising is typically best when the training and testing data noise levels coincide, though we observed some deviations from this rule for lower noise levels, where the quality of restoration of the NF spectrum also makes a noticeable contribution in the overall error value. When being trained on different noise levels, the NFT-Net was still able to produce denoising, thus demonstrating the design's flexibility. We have shown that conventional NFT calculation methods give ``distorted'' results when working added noise. In fact, the ``distorted'' results are actually correct, but from the nonlinear transformation point of view. But from the application's perspective, we are almost always interested in the denoised signals to reduce the embedded data corruption level.

Finally, we note that the problem of recovering a few solitons from a given pulse utilizing NN has been studied in~\cite{jgy18,ymm19,wxz20,mishina2021eigenvalue}. However, the NN architectures used in these works are much more simple as far as we need to identify and denoise just several solitonic parameters, while in our work we recovered 1024 complex numbers representing the continuous NF spectrum. Here we considered a larger number of solitary modes, where, however, only the total number of solitons in the pulse was studied. 
Potentially, it is interesting to combine the NN developed in our work with the additional module that can deal with soliton parameters restoration: such a hybrid tool would be able to perform the complete NFT decomposition of an arbitrary decaying pulse. However, from the position of high-speed optical transmission, the signal processing involving the continuous NF part is most important. We also anticipate that our results can find applications in numerous problems that involve the solution of direct scattering problems where denoising of the profiles is required.

To sum up, we investigated the modelling of the NFT operation associated with the focusing NLSE, using the NN with a special structure, which we coined the NFT-Net. We considered here an almost unexplored case dealing with the computation of the continuous part of the NF spectrum. It was demonstrated that the WaveNet-type NFT-Net structure can satisfactorily perform the task of the NF spectrum computation, and the best-performing architecture was obtained by Bayesian hyperparameters optimisation. Moreover, we showed that the same NFT-Net structure can be used to efficaciously retrieve both the reflection coefficient $r(\xi)$ and the scattering coefficient $b(\xi)$. The most practically important feature of the developed NN-based method is its capability to perform signal denoising.  We demonstrated that the NN-based processing can bring about essential improvements in the quality of NF spectrum restoration attributed to noise-perturbed time-domain profiles, compared to the conventional high-accuracy NFT processing method. The advantage in denoising becomes most pronounced at high noise levels, with the maximum restoration quality typically occurring when the SNR of the training data is the same as that of the validation dataset. 
In the end, we demonstrated that we can compute and denoise the NF b-coefficient with the same NFT-Net architecture, where the restoration quality follows the same trend as we observed for the recovery of the $r(\xi)$ coefficient. From the physical application's perspective, our methods can be useful in tasks where we are interested in separating the meaningful NF spectrum from the noise: it directly refers to the optical transmission applications and can be used in many similar problems. From the mathematical perspective, our work can be reckoned as a step towards developing artificial intelligence-based tools for the solution of integrable PDEs. We anticipate that the proposed approach can be extended well beyond optical communications, to signal processing, fibre Bragg grating design \cite{FBG01,FBG02} and other applications.

