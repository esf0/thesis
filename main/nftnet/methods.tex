\subsection{Complexity analysis}
One of the important metrics in the development of signal processing tools is the complexity of the processing device, i.e. the number of elementary arithmetic operations that the processing unit employs to reach its goal. 
Quite often we need to analyse the interplay between the complexity and accuracy of the processing unit. Thus, here we perform the complexity analysis for the NFT-Net.

In our case, we concentrate only on the number of multiplications, since in practical implementation the computational complexity of addition operations is negligible. 
The number of real multiplications needed for the forward propagation of the model, as introduced in \cite{freire2021performance} for several types of NN layers, is also used to calculate the computational complexity of the NFT-Net in this paper. 

The overall complexity $C$ of the NFT-Net can be presented as the sum of two constituents: the complexity of densely-connected block $C_{\text{dense}}$ and the complexity of convolutional block $C_{\text{conv}}$. 
For the calculation of $C_{\text{dense}}$ the same formula as in \cite{freire2021performance} can be used, where we have $n_i$ inputs, $n_1$ neurons in the hidden layers, and $n_o$ outputs, and the complexity is defined as:
\begin{equation}
C_{\text{dense}}=  n_1 \cdot (n_{i} + n_{o}) {,}
\label{eq:c_dense}
\end{equation}

In the case of the convolution layer, we can change the equation given in \cite{freire2021performance} to measure the generalised convolutional layer complexity by taking into account the number of filters $f$ and kernel size $k$, as well as the effect of padding $p$, stride $s$, and dilation $d$. The complexity $C_{\text{conv, layer}}$ for one layer when the input shape is [$L_{in},Q_{in}$], is specified as follows:
\begin{equation}
C_{\text{conv, layer}} = k \cdot Q_{in} \cdot f \cdot \left( \frac{L_{in} + 2\cdot p -d\cdot(k-1)-1}{s} +1\right) {,}
\label{eq:c_conv}
\end{equation}
where $Q_{in}$ denotes a number of channels, $L_{in}$ is a length of signal samples sequence.
Therefore, the total complexity of the NFT-Net used in this paper in terms of real multiplications per output sequence (1024 complex valued points) is:
\begin{equation}
C_{\text{conv}} = 2 \cdot (C_{\text{conv, 1}}+C_{\text{conv, 2}}+C_{\text{conv, 3}}+C_{\text{dense}}) {,}
\label{eq:c_total}
\end{equation}
where the factor 2 in front appears due to the use of two identical NNs to predict the real and imaginary parts of the continuous NF spectrum. Turning to our optimised architecture, to process  1024 complex signal samples, the following number of multiplication operations for the optimised architecture is required:
\begin{eqnarray}
    C_{\text{conv}} = 2\cdot(10\cdot2\cdot10\cdot1006+ 18\cdot10\cdot15 \cdot972 + 14\cdot15\cdot10\cdot320 + \nonumber \\ 
    + 3200\cdot4096 +4096\cdot1024) = 41598208 {.}
    \label{eq:c_total_num}
\end{eqnarray}