\cleardoublepage
\chapter*{Introduction}
\markboth{Introduction}{Introduction}
\addcontentsline{toc}{chapter}{Introduction}


\section*{Motivation}

% Optical fiber channels are pivotal in modern communication systems, enabling high-speed data transmission across the globe. They form the backbone of the internet and telecommunications infrastructure \cite{cambridge2022, aip2022}. The significance of optical fiber channels transcends beyond mere data transmission; they are the lifelines that facilitate global connectivity, drive economic growth, support social interactions, and are indispensable in various sectors including healthcare, finance, education, and government services. The evolution of the digital age, characterized by the proliferation of online services, the Internet of Things (IoT), and an ever-growing demand for higher bandwidth, accentuates the indispensable role of optical fiber channels. 

Optical fiber channels play a crucial role in today's communication systems, providing the high-speed data transmission worldwide. These channels are not just for data transfer—they are vital for maintaining global connections, contributing to economic development, and supporting essential services in healthcare, finance, education, and government. The rapid expansion of online services, the Internet of Things (IoT), and the increasing need for greater bandwidth highlight the essential role of optical fiber channels in the modern digital era \cite{cambridge2022, aip2022}.


% However, as the global data traffic surges, propelled by the advent of 5G/6G technologies, cloud computing, and an increasing number of connected devices, the existing optical fiber infrastructure is grappling with numerous challenges that threaten to impede its ability to meet the burgeoning data transmission requirements \cite{huawei2023}. These challenges encompass a wide spectrum, ranging from technical hurdles like capacity limitations and spectrum expansion to algorithmic and network planning complexities. The exigency of addressing these challenges is not only paramount for sustaining the current rate of digital proliferation but is also a requisite for fostering future innovations in the realm of communication technologies.

With the rise of new technologies like 5G/6G, cloud services, and more devices joining the internet, our current optical fiber networks face tough challenges to keep up with the growing demand for data transfer \cite{huawei2023}. These challenges include limits on how much data can be sent through and the need to expand the use of the available spectrum. They also involve complex algorithms and the planning of networks. Solving these problems is critical to keep up with the digital world's rapid growth and to enable new advancements in communication technologies.


% Various stakeholders, including academia, industry, and government bodies, are engrossed in concerted efforts to surmount these challenges. Research and development initiatives are being vigorously pursued to unravel new technologies and methodologies that can significantly augment the capacity, efficiency, and reliability of optical fiber channels \cite{engineering2023}. This includes exploring advanced modulation formats, developing next-generation fibers, innovating efficient network planning algorithms, and leveraging emerging technologies like machine learning and Nonlinear Fourier Transform (NFT) to optimize the performance of optical fiber communication systems \cite{Turitsyn:17, mdpi2023}. 

People from universities, companies, and governments are working hard to overcome the challenges faced by optical fiber networks. There is a strong push in research to find new ways to improve the ability of these networks to handle more data, work more efficiently, and be more reliable. Efforts include looking into better ways to encode data, creating new types of fibers, coming up with smarter network planning methods, and using new tools like machine learning and the \acrfull{nft} to get the most out of optical fiber communications \cite{turitsyn2017nonlinear}.


% \subsection*{Challenges}

% Optical fiber communication systems are confronted with a myriad of challenges, each embodying a set of complexities and potential solutions. A paramount challenge is enhancing the capacity of fiber channels, crucial for meeting the burgeoning demand for higher data rates. Improving spectral efficiency is a significant part of this endeavor, with advanced modulation formats like higher-order quadrature amplitude modulation (QAM) being explored for this purpose \cite{huawei2023}. Additionally, the expansion of available frequency bands beyond the conventional C-band, extending to L, S, E, and O bands, is anticipated to substantially augment transmission capacity \cite{huawei2023}. This expansion necessitates the development of fiber amplifiers that support these new spectral applications. Techniques and technologies such as forward error correction (FEC) and coherent detection are employed to improve the signal-to-noise ratio, enabling lower error rates over longer distances \cite{huawei2023}.

Optical fiber systems face many challenges, especially the need to increase their capacity to keep up with the growing demand for faster data speeds. One way to tackle this is to use more complex data encoding techniques, like advanced \gls{qam} \cite{huawei2023}. We're also looking to use more of the light spectrum for data transmission, which means we would use not just the usual C-band but also the L, S, E, and O bands. This requires making new types of amplifiers that work with these broader ranges of light. We use methods like \gls{fec} and technologies like coherent detection to make signals clearer and reduce data errors over long distances . 
Expanding the spectrum used in optical fiber communication involves several areas of development. One area is improving fiber amplifiers to work with a broader range of light frequencies, where technologies like rare-earth-doped fibers and semiconductor amplifiers are the key \cite{huawei2023}. Creating components that can operate at these new frequencies is essential to use the entire available spectrum and increase the amount of data transmitted. Alongside hardware, we also need better algorithms to manage these new frequencies efficiently. 

Another significant step is to create better fibers to overcome the limits of current ones. This includes making fibers with less loss and better handling of light interactions that can distort the signal. Efforts in this direction include researching fibers that lose less signal over distance and using \gls{sdm} to send more data through the same cable \cite{kaur2022recent}.


% The spectrum expansion challenge is multi-faceted, encompassing the development of fiber amplifiers for new spectral applications, with rare-earth-doped gain fibers and semiconductor optical amplifiers (SOAs) being among the key technologies explored \cite{huawei2023}. Manufacturing new high-frequency components is crucial for exploring new available frequency spectrums and expanding transmission capacity \cite{huawei2023}. Evolving transmission algorithms to efficiently utilize the expanded spectra is also part of this challenge.

% Transitioning to next-generation fibers that address the limitations of current single-mode optical fibers is imperative \cite{engineering2023}. Reducing intrinsic loss and enhancing anti-nonlinear effect ability are vital for the development of these fibers \cite{engineering2023}. Research on low-loss optical fibers and technologies related to space division multiplexing (SDM) are among the solutions proposed to tackle these challenges \cite{engineering2023}. Dispersion management techniques are employed to mitigate signal broadening, which limits data rate and reach. Overcoming higher-order-mode multipath interference (MPI) in fibers like G.654 is crucial for meeting transmission requirements across various frequency bands \cite{huawei2023}.



% Another significant step is to create better fibers to overcome the limits of current ones \cite{engineering2023}. This includes making fibers with less loss and better handling of light interactions that can distort the signal \cite{engineering2023}. Efforts in this direction include researching fibers that lose less signal over distance and using \gls{sdm} to send more data through the same cable \cite{engineering2023}. We also use strategies to control how light spreads out in the fiber to maintain high data rates over long distances. Addressing issues like interference from multiple light paths in advanced fibers is crucial for supporting data transmission in different light bands \cite{huawei2023}.



% Efficient network planning is essential in the emergence of ultra-large-scale optical networks. Efficient algorithms and strategies for Routing and Wavelength Assignment (RWA) are crucial for optimal network planning in these large-scale networks \cite{huawei2023}. Topology optimization for network expansion is essential for accommodating emerging data service applications \cite{huawei2023}.


% Algorithm development is critical in overcoming the nonlinearity challenges in optical channels. Nonlinear Fourier Transform (NFT) has been proposed to increase channel capacity beyond the Shannon limit, employing solitons as discrete outputs of the transform to carry information \cite{Turitsyn:17}.

% Nonlinear Fourier Transform (NFT) is one of the more recent methods explored for transmitting information in optical fiber communication systems, aiming to overcome the current state-of-the-art of multi-user transmission in wavelength division multiplexing (WDM) systems \cite{mdpi2023}. NFT based systems employ so-called "nonlinear modes", which evolve independently under the action of chromatic dispersion alone, similar to linear Fourier modes \cite{aps2023}. This method holds promise in mitigating nonlinear signal distortion in optical channels, which is one of the significant challenges in optical fiber communication systems \cite{aps2023}.

Developing algorithms to handle nonlinearity in optical channels is essential. The \gls{nft} is a mathematical tool that extends the concept of the classical Fourier transform to include the effects of nonlinearity in signal processing. In optical communication systems, nonlinearity can significantly impact signal propagation in optical fibers, often leading to distortion that can limit the performance and capacity of these systems.~\cite{turitsyn2017nonlinear}.
\acrshort{nft} takes into account the nonlinear nature of the optical fiber medium. By analyzing the signal in a nonlinear spectral domain, it can effectively describe the evolution of complex waveforms that are affected by the fiber's nonlinearity, thereby mitigating distortion.
Traditional methods of communication approach the Shannon limit because they are based on linear signal processing. \acrshort{nft} offers a way to possibly exceed this limit by using the fiber's nonlinearity to our advantage rather than treating it as a hindrance. Moreover, \gls{nft} naturally leads to the concept of solitons~\cite{hasegawa1973}, which are stable wave packets that can travel over long distances without changing shape. \acrshort{nft}-based systems can use solitons to carry information more reliably over long distances. \gls{nft} can be used to improve signal processing algorithms. By transforming the data into the nonlinear domain, \acrshort{nft} allows for the development of new signal processing algorithms that are better suited to dealing with the specific types of distortions and noise that occur in optical fibers. This can lead to improved spectral efficiency by allowing more data to be packed into the same bandwidth. 

% NFT-based systems utilize "nonlinear modes" that behave similarly to linear Fourier modes but are only influenced by chromatic dispersion \cite{aps2023}. This approach is promising for reducing signal distortion caused by nonlinearity.



% \subsection*{Technological Advancements}

% The importance of resolving the challenges associated with optical fiber communication systems is underscored by the integral role they play in modern communications. As the backbone of today's global economy and the information age, these systems offer a plethora of advantages over traditional copper cables such as faster signal transmission, reduced attenuation over long distances, and minimized electrical interference \cite{cambridge2022, aip2022}. The ability to transmit data at high speeds with lower loss makes optical fiber channels the linchpin of our connected world, supporting various sectors and facilitating global interactions. 

% However, the journey towards optimizing the performance of optical fiber communication systems necessitates a thorough examination and implementation of cutting-edge technological advancements. One such advancement is Spatial Division Multiplexing (SDM), which holds the potential to significantly augment the capacity of optical fiber channels by facilitating the transmission of multiple data streams within a single optical fiber \cite{engineering2023}. This technology is seminal in harnessing the full capacity potential of optical fibers, thereby meeting the escalating data transmission demands.

% Moreover, the advent of advanced modulation formats is pivotal in enhancing spectral efficiency, which in turn, escalates the capacity of optical fiber channels \cite{huawei2023}. These modulation formats, including higher-order quadrature amplitude modulation (QAM), are instrumental in efficiently utilizing the available spectrum, thereby maximizing data throughput.

% The exploration of new fiber types, such as hollow-core fibers and the ongoing research on low-loss optical fibers, also constitute a significant stride towards overcoming the inherent challenges of loss and nonlinearity \cite{engineering2023}. These novel fiber types are designed to mitigate signal attenuation and nonlinear distortions, which are detrimental to the performance of optical fiber communication systems.

% Machine learning (ML) algorithms are increasingly being recognized for their capability to optimize the performance of optical fiber communication systems \cite{ieee_ml_2022}. By leveraging ML, it is feasible to develop robust solutions for addressing nonlinear distortions and other signal impairments, which are perennial challenges in optical fiber communication. ML algorithms can autonomously monitor, diagnose, and rectify network anomalies, thereby enhancing the overall reliability and efficiency of optical fiber channels.

% The fusion of these technological advancements is instrumental in propelling optical fiber communication systems to new horizons, ensuring they remain robust and capable of catering to the ever-evolving data transmission requirements. By harnessing these technologies, stakeholders can significantly ameliorate the challenges associated with optical fiber communication systems, thereby fostering a conducive ecosystem for the continuous growth and evolution of the global digital society.

% Furthermore, innovations in optical amplifiers and switches are pivotal in alleviating challenges related to attenuation and loss \cite{engineering2023}. By improving the efficiency of optical amplifiers and switches, it is possible to extend the reach of optical signals and enhance the performance of optical fiber channels.


Machine learning (ML) is becoming an essential tool for optical communication systems due to its ability to understand complex patterns and make data-driven decisions. In the field of optical communications, where signals can be distorted by a variety of factors such as fiber nonlinearity and channel noise, ML stands out as a solution that can adaptively compensate for these impairments without explicit programming. This adaptability is crucial, as it allows communication systems to self-optimize in real-time, ensuring the transmission of data with higher accuracy and speed. Moreover, ML's predictive capabilities enable proactive maintenance, reducing downtime and enhancing the reliability of communication networks.

The utilization of ML in optical communication systems is not just about solving current challenges; it's also about paving the way for future advancements. As data traffic volumes continue to grow, ML provides a scalable approach to managing this increase efficiently. By analyzing vast amounts of transmission data, ML algorithms can predict and manage network loads, optimize routing, and even aid in the design of new fiber optic materials and structures. This foresight is invaluable for keeping communication systems ahead of the curve, making ML not just a tool for maintenance but also a driving force for innovation in the field of optical communications.



% Machine learning (ML) methods are now key to improving how optical fiber communication systems work \cite{ieee_ml_2022}. These methods can help solve problems caused by nonlinear distortions and other issues in transmitting signals through optical fibers. ML can automatically keep track of the network's condition, identify problems, and fix them, which makes the fiber optic network more reliable and efficient.

Bringing together these advancements is key to pushing optical fiber communications forward. These technologies ensure that these systems are strong and ready for the growing demands of data transmission. With these tools, we can improve how optical fiber systems work, supporting the ongoing development and growth of our digital world.


\section*{Applications in Diverse Fields}

\acrlong{nlse} and the Manakov system have revolutionized our understanding of nonlinear wave dynamics in various physical settings. These mathematical models capture the essence of wave propagation influenced by nonlinear and dispersive effects, forming the backbone of numerous technological advancements and scientific discoveries~\cite{karjanto2019nonlinear}. The \acrlong{nft} stands out as a transformative tool, similar to the classical Fourier Transform but for nonlinear systems, enabling the decomposition of complex wave fields into simpler, analyzable components.

Consider the Bose-Einstein Condensates (BECs), where the NLS equation, known as the Gross-Pitaevskii equation, provides insights into the macroscopic wave function at zero temperature. Efforts to solve the GPE analytically for ultracold atom clouds in the BEC phase have utilized variational techniques, offering insights into the dynamics of these systems \cite{perez1996low,perez1997dynamics}. The interplay between kinetic energy, trapping potential, and particle interactions is finely detailed, paving the way for the exploration of quantum effects such as soliton interactions and vortex formations. The \acrshort{nft}'s power is magnified by \acrlong{nn}s, which can predict the evolution of these phenomena from empirical data, even when the system's complexity exceeds traditional analytical capabilities. For instance, \acrlong{nn}s could forecast the dynamics of solitons in BECs, providing a new lens through which quantum turbulence and particle interactions are viewed.

In plasma physics and water wave studies, the \acrshort{nlse} models the behavior of envelope solitons, essential for grasping Langmuir waves and modulational instability. It also aids in understanding the genesis of rogue waves and their modulation on the ocean's surface. 
The spatial \acrshort{nlse}'s utility in modeling deterministic freak wave generation in water at MARIN illustrates its practical application, guided by the theory of linear perturbation analysis and solitons on a non-vanishing background \cite{van2006displaced,karjanto2006mathematical,karjanto2007extreme}. The \acrshort{nlse} has been used as a framework for studying the emergence of rogue waves in the ocean, attributed to nonlinear energy transfer processes, through both deterministic and stochastic approaches \cite{henderson1999unsteady,pelinovsky2008extreme,kharif2008rogue}.
Here, \acrshort{nft}, coupled with \acrlong{nn} analysis, unravels the nonlinear spectral dynamics, facilitating the prediction of wave behavior under diverse conditions—a crucial step for designing plasma-based devices or mitigating rogue wave impacts. 

The Nonlinear Fourier Transform offers a novel approach to analyzing complex wave dynamics in superconductivity, building on the foundational work with the nonlinear Klein-Gordon and sine-Gordon equations. These equations help us understand how waves move and interact in superconductors and have been crucial in studying phenomena like crystal dislocation and particle interactions \cite{dodd1982solitons, drazin1989solitons, frenkel1939theory, perring1962model, gibbon1979example}.
A key area where \acrshort{nft} shows promise is in examining the Josephson effect, observed in Josephson junctions—quantum devices made of superconductor electrodes separated by a barrier. This effect, where current flows without voltage, highlights the quantum mechanical nature of superconductivity \cite{josephson1962possible,josephson1974discovery}. The sine-Gordon model, used to describe Josephson junctions, aligns well with \acrshort{nft}'s capabilities, especially in analyzing superconducting soliton oscillators \cite{parmentier1993solitons}.
Further, the connection between the sine-Gordon system and the \acrshort{nlse} in small-amplitude scenarios opens up possibilities for using \acrshort{nft} to study superconductors under external influences, like alternating current fields. This is where \acrlong{nn}s come into play. By performing \acrshort{nft}, \acrlong{nn}s can analyze complex superconducting systems, predict their behavior, and identify patterns that are not immediately apparent \cite{kaup1978theory, gro1993nonlinear}.
Recent advancements in quantum graph theory illustrate how \acrshort{nft}, coupled with \acrlong{nn}s, can be particularly effective in superconductivity research. For example, studying localized wave solutions in tricrystal Josephson junctions showcases the potential of \acrlong{nn}s to understand the interactions within superconducting materials \cite{gnutzmann2006quantum,kuchment2008quantum,berkolaiko2013introduction,susanto2019soliton}.

Nonlinear optics, another area enriched by the \acrshort{nlse}, sees the formation of spatial and temporal solitons in nonlinear birefringent media. Optical rogue waves, modeled by a generalized \acrshort{nlse} and supported by experimental data from highly nonlinear microstructured optical fibers, exemplify the fusion of theory and practice in this field \cite{solli2007optical,akhmediev2013recent,dudley2008harnessing,bonatto2011deterministic}. The introduction of \acrlong{nn}s to perform \acrshort{nft} provides a groundbreaking method for analyzing light-matter interactions, facilitating the development of advanced optical devices. In quantum fluids, \acrshort{nft} aids in understanding superfluid turbulence and vortex interactions, crucial for quantum computing and simulation technologies.

The NLS equation's application extends to atmospheric sciences, modeling nonlinear wave propagation including internal waves and solitons in stratified fluids. Leveraging \acrshort{nft} with \acrlong{nn}s allows for enhanced weather prediction models and a deeper understanding of climate dynamics.


The methods from my thesis, like using \acrlong{nn}s and a window approach, offer new ways to study and predict complex system behaviors. These tools are great for cases where we might not know the exact math or when systems don't fully match the usual conditions for \acrshort{nft}. Neural networks can find new patterns in data, helping us understand phenomena like rogue waves better. The window approach lets us study complex systems more easily, leading to new research and technologies in various fields, from environmental science to quantum tech and communications.
Using \acrshort{nlse} and \acrshort{nft} methods in areas beyond optical communications shows how versatile these approaches are for studying wave dynamics. This research not only adds to academic knowledge but also opens up possibilities for new technologies and applications in many physical systems.


\section*{\acrlong{nft} in Systems with Higher-Order Effects}


Integrable systems, characterized by an infinite number of conservation laws, are idealized models where soliton solutions can propagate without changing shape due to exact balance between nonlinearity and dispersion. The \acrshort{nft} provides a powerful framework for analyzing these systems by transforming the \acrlong{nlse}, a model governing soliton dynamics in fiber optics and other fields, into a spectral problem. This transformation enables the study of the system's evolution in terms of its spectral data, simplifying the analysis of soliton interactions and stability.

However, real-world systems are rarely perfectly integrable due to higher-order effects like attenuation, higher-order dispersion, and nonlinearity. These perturbations introduce deviations from the ideal soliton dynamics predicted by the \acrshort{nlse}~\cite{karpman1977perturbation}. For instance, higher-order dispersion can lead to the emission of radiation during soliton propagation, altering the soliton's shape and velocity. Similarly, higher-order nonlinearity can cause soliton self-frequency shift, a phenomenon where the central frequency of a soliton shifts during propagation.

Incorporating these higher-order effects into the \acrshort{nft} analysis requires perturbative techniques. One approach is to treat these effects as small perturbations to the integrable system and study their impact on the spectral data. For example, the effect of perturbations on soliton parameters (such as amplitude, width, and velocity) can be analyzed by examining how the spectral data evolves due to these nonidealities \cite{kraych2019statistical,gelash2019bound,chowdury2018modulation}. This perturbative analysis helps in understanding the stability of solitons under real-world conditions and in designing optical systems that can mitigate the adverse effects of higher-order corrections.

A concrete example of this analysis is seen in the study \cite{kivshar1989dynamics} of fiber optics, where solitons are used for long-distance communication. The presence of higher-order dispersion and nonlinearity in optical fibers affects the soliton's shape and speed, leading to signal distortion. By applying \acrshort{nft} with perturbative corrections \cite{kaup1976perturbation}, engineers can predict these changes and design optical systems that compensate for these effects, ensuring stable soliton propagation and reliable communication.


In the domain of soliton dynamics within nearly integrable systems, a significant body of research has illuminated the intricate effects of higher-order corrections. The work of Kodama and Hasegawa in 1982 \cite{kodama1982amplification}, for example, delves into the realm of optical solitons in monomode fibers, focusing on how dissipative damping can be counteracted through periodically applied stimulated Raman scattering. This study stands as a cornerstone in understanding the interplay between soliton stability and external perturbations.

Further extending the discourse \cite{dianov1985nonlinear} embarked on a numerical exploration of a compensation model grounded in the application of a small-amplitude Raman pumping wave. Their findings offer a deeper comprehension of how external forces can mold soliton dynamics, specifically highlighting the potential for dissipation to be offset in a controlled manner.

The analytical frameworks provided by Kaup and Newell in the late 1970s \cite{kaup1978solitons,kaup1978theory} offer another layer of insight into the perturbed soliton solutions. Their work, through a dynamical systems lens, sheds light on the resilience and behavioral tendencies of solitons when subjected to external perturbations, enriching the dialogue on soliton dynamics with analytical rigor.

For a deeper dive into the latest studies, some numerous contemporary articles and findings shed light on advancements in this field. The article \cite{peleg2020radiation} investigates the emission of radiation during fast collisions between two solitons governed by the nonlinear Schrödinger equation with weak cubic loss. Employing perturbation theory and numerical simulations, the study reveals the impact of cubic loss on soliton dynamics, emphasizing the balance between conservative and dissipative perturbations in shaping radiation outcomes. This research contributes to the broader understanding of soliton interactions in nonlinear optical systems. The article \cite{hause2018impact} explores the effects of fiber optic loss on the dynamics of two-soliton states in nonlinear Schrödinger equations. It particularly focuses on how loss modifies the eigenvalue spectrum of soliton compounds, leading to significant changes in their behavior. This investigation reveals that fiber loss can enhance one soliton at the expense of another and transform static solitons into pairs with distinct velocities, providing new insights into soliton dynamics and their application in fiber-optic communications.

These studies collectively contribute to a nuanced understanding of how higher-order effects, ranging from external perturbations to dissipation, shape the evolution of solitons in nearly integrable systems. The interplay between theory and application revealed through these works underscores the dynamic nature of soliton research and its relevance to fields as diverse as optical communications and fluid dynamics. Through a careful examination of these contributions, one can appreciate the breadth of inquiry and the depth of discovery that characterize the ongoing exploration of soliton dynamics in the face of higher-order corrections.

Another book "Nonlinear Waves in Integrable and Nonintegrable Systems" by Jianke Yang \cite{yang2010nonlinear} explores the mathematical and computational approaches to understanding nonlinear waves in both integrable and nonintegrable systems. It addresses significant problems in physical, life sciences, and engineering through analytical and computational techniques, promoting an interdisciplinary approach. The book covers various topics including the derivation of nonlinear wave equations, integrable theory, soliton perturbation theories, theories for nonintegrable equations, nonlinear wave phenomena in periodic media, and numerical methods for studying nonlinear wave equations.

The Nonlinear Fourier Transform emerges as a robust tool for analyzing integrable systems even when higher-order effects—such as attenuation and higher-order dispersion—come into play. Through the detailed studies and examples provided, it's evident that \acrshort{nft} can effectively account for these perturbations, enabling a more accurate analysis of soliton dynamics and interactions in real-world systems. By extending traditional \acrshort{nft} techniques with perturbation theory, researchers can predict and mitigate the impact of these non-idealities, ensuring the stability and reliability of soliton-based technologies in various applications, from optical communications to quantum computing.


\section*{Contribution}
The contribution of this thesis can be summarized as follows:
% \begin{itemize}
%     \item New approach to use \acrshort{nft} for continuous signals was proposed and developped. It consist of using \acrfull{cdc} for preprocessing of the signal and subsequent \acrshort{nft} processing. For single polarisation results show uotperform of \acrfull{dbp} with 3 steps per span, and for two polarisations the results on the level of \acrshort{dbp} with 2 steps per span. The code was developped and available online.
%     \item WDM and OFDM symbols was analysed using \gls{nft} to show conditions of presense of solitons. This analysis continued with analysis of WDM signals showing that dominant amount (99\%) of energy corresponds to soliton components.
%     \item New approach to perform forward and inverse \acrshort{nft} using Neural networks was introduced. Selected architecture was additionally optimised and shows capability to work with noise signals while determenistic \acrshort{nft} algorithms fail.
%     \item New GPU accelerated framework (\gls{hpcom}) to simulate optical fibre channels with arbitrary characteristics was introduced. This framework used as data-mining tool for following research. Code examples are provided with the example results and metrics.
%     \item Gradient boosting was introduced as a promising tool for nonlinear equalisation.
%     \item Example of performance of \gls{nn}s for different range of channel and signal parameters was shown.
%     \item Data-driven approach was used to reveal new complex structures in received symbols constellation which is the result of optical fibre nonlinearity. The non-gaussian structure showed for all ``triplets'' of received points.
%     \item In addition: the demonstration for optical channel simulation in real time is provided.
% \end{itemize}
\begin{itemize}
    \item We proposed and developed a new method that applies the \acrshort{nft} to continuous signals. This method involves pre-processing the signal using \acrfull{cdc} before applying \acrshort{nft}. The outcomes for single polarization signals have shown better performance than \acrfull{dbp} with three steps per span, and for dual polarization, the results are compared with \acrshort{dbp} with two steps per span. We have made the developed code publicly available.
    
    \item Through the use of \acrshort{nft}, we analyzed \gls{wdm} and \gls{ofdm} symbols to determine the conditions for the presence of solitons. Our analysis of \gls{wdm} signals revealed that a dominant portion of the energy (99\%) is associated with soliton components.
    
    \item We introduced a new approach for conducting both forward and inverse \acrshort{nft} operations using \acrshort{nn}s. The chosen \acrshort{nn} architecture was further optimized and has demonstrated effectiveness in working with noisy signals, where deterministic \acrshort{nft} algorithms were unsuccessful.
    
    \item A new GPU-accelerated framework, \acrfull{hpcom}, was introduced for simulating optical fiber channels with various characteristics. This framework serves as a data mining tool for subsequent research, and we have provided code examples along with the resulting data and performance metrics.
    
    \item We identified \gls{gb} as a promising technique for nonlinear equalization within optical channels.
    
    \item We demonstrated the performance of \acrshort{nn}s across a range of channel and signal parameters, showcasing their versatility and effectiveness.
    
    \item A data-driven approach was employed to uncover new complex structures within the received symbols' constellation, which arise from the nonlinearity of optical fibers. This non-Gaussian structure was evident across all examined ``triplets'' of received points.
    
    \item Furthermore, we provided a demonstration of real-time optical channel simulation as an additional resource.
\end{itemize}



\section*{Thesis outline}
This thesis is organised as follows.

The thesis divided into two parts: ``Advanced Nonlinear Fourier Transform Techniques'' and ``Machine Learning for Optimizing Optical Communication Systems''.

In the first part
\begin{itemize}
    \item Chapter 1 provides a comprehensive introduction to the Nonlinear Fourier Transform (\acrshort{nft}) and its application in modeling optical fiber channels for both single and dual-polarization systems. The chapter further discusses various modulation formats and the performance metrics relevant to this study, concluding with an in-depth look at Wavelength Division Multiplexing (\acrshort{wdm}) and Orthogonal Frequency-Division Multiplexing (\acrshort{ofdm}), which are central to the research conducted in this thesis.
    
    \item Chapter 2 delves into the processing of optical signals using a sliding window technique and introduces an enhanced method combining Chromatic Dispersion Compensation (\acrshort{cdc}) with \acrshort{nft}. Results for \acrshort{wdm} signals in both single and dual-polarization systems are presented at the end of this chapter.
    
    \item Chapter 3 explores the solitonic components of optical signals, beginning with the establishment of criteria for energy and power levels required for soliton existence. This is followed by an analysis of individual \acrshort{ofdm} and \acrshort{wdm} symbols to assess the probability of soliton occurrence based on symbol parameters. The chapter proceeds with an investigation into the presence of solitons in continuous \acrshort{wdm} signals and the distribution of their properties.
    
    \item Chapter 4 wraps up the first part of the thesis by introducing the application of Neural Networks (\acrshort{nn}) for \acrshort{nft}. It opens with the motivation behind using \acrshort{nn} and demonstrates that \acrshort{cnn}s can effectively compute \acrshort{nft}. The approach is further validated for its ability to identify solitonic components within a signal. The final section of this chapter provides detailed findings on the optimal structure for \acrshort{nn} and its capability to manage \acrshort{nft} in the presence of noise.
\end{itemize}
% \begin{itemize}
%     \item Chapter 1 are devoted to \acrshort{nft} fundamentals and describes optical fibre channel models, theory of \acrshort{nft} for models with one and two polarisations. This chapter continues with modulation formats and system performance metrics, used in this thesis. The chapter concludes with mathematical foundations of \acrfull{wdm} and \acrfull{ofdm} types of signal, which are used in researches in this thesis.
%     \item Chapter 2 starts with sliding window approach of processing optical signals. The new approach which use \acrfull{cdc} for enhancing \acrshort{nft} processing after introduced. The chapter concludes with the results for \acrshort{wdm} signal for one and two polarisation systems.
%     \item Chapter 3 expore the analysis of solitonic component of optical signals. It starts with criteria which determin energy and power levels for possibility of soliton existance. After that the analysis of single \acrshort{ofdm} and \acrshort{wdm} symbol for probability of soliton occurence depending on symbol parameters was done. The chapter continues with the analysis of continuous \acrshort{wdm} signal for the precense of solitons and properties of their distribution.
%     \item Chapter 4 concludes the first part and introduce \acrfull{nn} approach to perform \acrshort{nft}. It starts with motivation of usage of \acrshort{nn} and after shows that convolution based \acrshort{nn} can successfully perform \acrshort{nft}. Next part shows that same approach can be used to detect solitonic component of the signal. The chapter concludes with extensive finding of optimal structure for \acrshort{nn} and it's ability to perform \acrshort{nft} for signals with additional noise.
% \end{itemize}

% The work presented in this thesis is structured into several chapters, each addressing key aspects of optical communication systems:

In the second part
% \begin{itemize}
%     \item Chapter 5 introduces \acrfull{hpcom} -- a data-mining simulation tool for optical channel models. The chapter continues with description of methodology beyond the simulations, after it shows framework architecture and main modules. Chapter continues with showing of performance boost in simulation time and concludes with several examples of optical communication system simulations.
%     \item Chapter 6 starts with introduction for \acrfull{gb} for nonlinearity equalisation in optical communication systems. It continues with theory of gradient boosting, after that chapter describes the methodology of application for practical systems and describe complexity of the \acrshort{gb} in general case. After, chapter shows results of \acrshort{gb} performance for different ranges of signal powers and propagation distances. Chapter concludes with another example of using machine learning aprroaches for equalisation -- \acrfull{nn}.
%     \item Chapter 7 itroduces the concept of analysis of received symbols disctibution. It showes what possible distribution types can be used for preliminary analysis and continues with results of using \acrfull{gmm} for describing and simulating received constellation distribution. 
%     \item Chapter 8 concludes the thesis and gives outlook for future directions of research.
% \end{itemize}
\begin{itemize}
    \item Chapter 5 presents the \acrfull{hpcom} framework, a new tool for simulating optical channels. It begins with an overview of the simulation methods, discusses the architecture of the framework, and demonstrates how it can speed up simulations. The chapter concludes with examples demonstrating the framework's application in simulating optical communication systems.
    
    \item Chapter 6 introduces the use of \acrfull{gb} for nonlinear equalisation in optical communication systems. The chapter progresses from the theoretical basis of gradient boosting to its practical application in complex systems, analyzing its computational complexity. Results showcase the efficacy of \acrshort{gb} across various signal powers and transmission distances, and the chapter ends with a example of the broader use of machine learning methods like \acrfull{nn} for signal equalization tasks.
    
    \item Chapter 7 delves into the study of received symbol distributions in optical communication systems, outlining potential distribution types for initial analysis. It elaborates on employing the \acrfull{gmm} to describe and emulate the distribution of the received constellation.
    
    \item Chapter 8 wraps up the thesis with a summary of the findings and provides insights into future research directions within the field of optical communications.
\end{itemize}
