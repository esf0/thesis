\cleardoublepage
\chapter*{Introduction}
\markboth{Introduction}{Introduction}
\addcontentsline{toc}{chapter}{Introduction}


\section*{Motivation}

Optical fiber channels are pivotal in modern communication systems, enabling high-speed data transmission across the globe. They form the backbone of the internet and telecommunications infrastructure \cite{cambridge2022, aip2022}. The significance of optical fiber channels transcends beyond mere data transmission; they are the lifelines that facilitate global connectivity, drive economic growth, support social interactions, and are indispensable in various sectors including healthcare, finance, education, and government services. The evolution of the digital age, characterized by the proliferation of online services, the Internet of Things (IoT), and an ever-growing demand for higher bandwidth, accentuates the indispensable role of optical fiber channels. 

However, as the global data traffic surges, propelled by the advent of 5G/6G technologies, cloud computing, and an increasing number of connected devices, the existing optical fiber infrastructure is grappling with numerous challenges that threaten to impede its ability to meet the burgeoning data transmission requirements \cite{huawei2023}. These challenges encompass a wide spectrum, ranging from technical hurdles like capacity limitations and spectrum expansion to algorithmic and network planning complexities. The exigency of addressing these challenges is not only paramount for sustaining the current rate of digital proliferation but is also a requisite for fostering future innovations in the realm of communication technologies.

Various stakeholders, including academia, industry, and government bodies, are engrossed in concerted efforts to surmount these challenges. Research and development initiatives are being vigorously pursued to unravel new technologies and methodologies that can significantly augment the capacity, efficiency, and reliability of optical fiber channels \cite{engineering2023}. This includes exploring advanced modulation formats, developing next-generation fibers, innovating efficient network planning algorithms, and leveraging emerging technologies like machine learning and Nonlinear Fourier Transform (NFT) to optimize the performance of optical fiber communication systems \cite{nature2023, mdpi2023}. 

\subsection*{Challenges}

Optical fiber communication systems are confronted with a myriad of challenges, each embodying a set of complexities and potential solutions. A paramount challenge is enhancing the capacity of fiber channels, crucial for meeting the burgeoning demand for higher data rates. Improving spectral efficiency is a significant part of this endeavor, with advanced modulation formats like higher-order quadrature amplitude modulation (QAM) being explored for this purpose \cite{huawei2023}. Additionally, the expansion of available frequency bands beyond the conventional C-band, extending to L, S, E, and O bands, is anticipated to substantially augment transmission capacity \cite{huawei2023}. This expansion necessitates the development of fiber amplifiers that support these new spectral applications. Techniques and technologies such as forward error correction (FEC) and coherent detection are employed to improve the signal-to-noise ratio, enabling lower error rates over longer distances \cite{huawei2023}.

The spectrum expansion challenge is multi-faceted, encompassing the development of fiber amplifiers for new spectral applications, with rare-earth-doped gain fibers and semiconductor optical amplifiers (SOAs) being among the key technologies explored \cite{huawei2023}. Manufacturing new high-frequency components is crucial for exploring new available frequency spectrums and expanding transmission capacity \cite{huawei2023}. Evolving transmission algorithms to efficiently utilize the expanded spectra is also part of this challenge.

Transitioning to next-generation fibers that address the limitations of current single-mode optical fibers is imperative \cite{engineering2023}. Reducing intrinsic loss and enhancing anti-nonlinear effect ability are vital for the development of these fibers \cite{engineering2023}. Research on low-loss optical fibers and technologies related to space division multiplexing (SDM) are among the solutions proposed to tackle these challenges \cite{engineering2023}. Dispersion management techniques are employed to mitigate signal broadening, which limits data rate and reach. Overcoming higher-order-mode multipath interference (MPI) in fibers like G.654 is crucial for meeting transmission requirements across various frequency bands \cite{huawei2023}.

Algorithm development is critical in overcoming the nonlinearity challenges in optical channels. Nonlinear Fourier Transform (NFT) has been proposed to increase channel capacity beyond the Shannon limit, employing solitons as discrete outputs of the transform to carry information \cite{nature2023}.

Efficient network planning is essential in the emergence of ultra-large-scale optical networks. Efficient algorithms and strategies for Routing and Wavelength Assignment (RWA) are crucial for optimal network planning in these large-scale networks \cite{huawei2023}. Topology optimization for network expansion is essential for accommodating emerging data service applications \cite{huawei2023}.

Nonlinear Fourier Transform (NFT) is one of the more recent methods explored for transmitting information in optical fiber communication systems, aiming to overcome the current state-of-the-art of multi-user transmission in wavelength division multiplexing (WDM) systems \cite{mdpi2023}. NFT based systems employ so-called "nonlinear modes", which evolve independently under the action of chromatic dispersion alone, similar to linear Fourier modes \cite{aps2023}. This method holds promise in mitigating nonlinear signal distortion in optical channels, which is one of the significant challenges in optical fiber communication systems \cite{aps2023}.

\subsection*{Technological Advancements}

The importance of resolving the challenges associated with optical fiber communication systems is underscored by the integral role they play in modern communications. As the backbone of today's global economy and the information age, these systems offer a plethora of advantages over traditional copper cables such as faster signal transmission, reduced attenuation over long distances, and minimized electrical interference \cite{cambridge2022, aip2022}. The ability to transmit data at high speeds with lower loss makes optical fiber channels the linchpin of our connected world, supporting various sectors and facilitating global interactions. 

However, the journey towards optimizing the performance of optical fiber communication systems necessitates a thorough examination and implementation of cutting-edge technological advancements. One such advancement is Spatial Division Multiplexing (SDM), which holds the potential to significantly augment the capacity of optical fiber channels by facilitating the transmission of multiple data streams within a single optical fiber \cite{engineering2023}. This technology is seminal in harnessing the full capacity potential of optical fibers, thereby meeting the escalating data transmission demands.

Moreover, the advent of advanced modulation formats is pivotal in enhancing spectral efficiency, which in turn, escalates the capacity of optical fiber channels \cite{huawei2023}. These modulation formats, including higher-order quadrature amplitude modulation (QAM), are instrumental in efficiently utilizing the available spectrum, thereby maximizing data throughput.

The exploration of new fiber types, such as hollow-core fibers and the ongoing research on low-loss optical fibers, also constitute a significant stride towards overcoming the inherent challenges of loss and nonlinearity \cite{engineering2023}. These novel fiber types are designed to mitigate signal attenuation and nonlinear distortions, which are detrimental to the performance of optical fiber communication systems.

Machine learning (ML) algorithms are increasingly being recognized for their capability to optimize the performance of optical fiber communication systems \cite{ieee_ml_2022}. By leveraging ML, it is feasible to develop robust solutions for addressing nonlinear distortions and other signal impairments, which are perennial challenges in optical fiber communication. ML algorithms can autonomously monitor, diagnose, and rectify network anomalies, thereby enhancing the overall reliability and efficiency of optical fiber channels.

Furthermore, innovations in optical amplifiers and switches are pivotal in alleviating challenges related to attenuation and loss \cite{engineering2023}. By improving the efficiency of optical amplifiers and switches, it is possible to extend the reach of optical signals and enhance the performance of optical fiber channels.

The fusion of these technological advancements is instrumental in propelling optical fiber communication systems to new horizons, ensuring they remain robust and capable of catering to the ever-evolving data transmission requirements. By harnessing these technologies, stakeholders can significantly ameliorate the challenges associated with optical fiber communication systems, thereby fostering a conducive ecosystem for the continuous growth and evolution of the global digital society.





\section*{Contribution}

\lipsum[1]

\section*{Thesis outline}

\lipsum[1]

