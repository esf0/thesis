\subsection{Genetic Algorithm Operations}

\subsubsection{1. Fitness Evaluation:}
The fitness function quantifies how close a given solution is to the optimum. It's a fundamental concept in GA as it guides the algorithm towards the solution.
\[
f(x) : \mathcal{D} \rightarrow \mathbb{R}
\]
where \( x \) is an individual in the domain \( \mathcal{D} \) of possible solutions, and \( f(x) \) is the fitness value of \( x \).

\subsubsection{2. Selection:}
Tournament Selection: In this method, \( k \) individuals are selected randomly from the population, and the one with the highest fitness is chosen. This process is repeated to create a mating pool.
\[
S(i) = \frac{1}{k} \cdot \max(f(i))
\]

\subsubsection{3. Crossover (Recombination):}
Two-Point Crossover: Two crossover points are selected randomly in the parent chromosome, and the genes between these points are interchanged to create new offspring.
For parents \( P_1 \) and \( P_2 \), and crossover points \( c_1 \) and \( c_2 \), the offspring \( O_1 \) and \( O_2 \) are generated as:
\[
O_1 = P_1[0:c_1] + P_2[c_1:c_2] + P_1[c_2:\text{end}]
\]
\[
O_2 = P_2[0:c_1] + P_1[c_1:c_2] + P_2[c_2:\text{end}]
\]

The crossover operation in Genetic Algorithms (GAs) is a fundamental process that combines the genetic material of two parent individuals to produce one or more offspring. The probability of crossover occurring is often denoted by  \( p_c \) or, as in code, CXPB. The process can be summarized by the following steps:

\begin{itemize}
    \item For each pair of parent individuals \( P_1 \) and \( P_2 \), a random number \( r \) is generated uniformly from the interval [0, 1].
    \item If \( r < p_c \) (where \( p_c \) is the crossover probability), crossover is performed to generate offspring \( O_1 \) and \( O_2 \).
    \item If \( r \geq p_c \), no crossover is performed and the parents are copied to the next generation as they are.
\end{itemize}

The offspring \( O_1 \) and \( O_2 \) are generated as follows for two-point crossover:
\[
O_1 = P_1[0:c_1] + P_2[c_1:c_2] + P_1[c_2:\text{end}]
\]
\[
O_2 = P_2[0:c_1] + P_1[c_1:c_2] + P_2[c_2:\text{end}]
\]

The probability of crossover occurring is represented as:
\[
p_c = \text{CXPB} = 0.5
\]

In this scenario, the crossover operation is conditional on a random event, and the crossover probability \( p_c \) (or CXPB) determines the likelihood of this event.

\subsubsection{4. Mutation:}
Gaussian Mutation: A value from a Gaussian distribution is added to the attribute value of the individuals.
For an individual \( x \) with attribute \( x_i \), the mutated attribute \( x_i' \) is:
\[
x_i' = x_i + \mathcal{N}(\mu, \sigma)
\]
where \( \mathcal{N}(\mu, \sigma) \) is a random value from a Gaussian distribution with mean \( \mu \) and standard deviation \( \sigma \).

\subsubsection{5. Elitism:}
Elitism ensures that a certain number of the best individuals from the current population are carried over to the next population. This helps to preserve the best found solutions.
For a population \( P \) and elitism size \( E \), the set of elite individuals \( \mathcal{E} \) is:
\[
\mathcal{E} = \{ x \in P | f(x) \geq f(x') \, \forall \, x' \in P \}
\]
where \( |\mathcal{E}| = E \).

\subsubsection{6. Population Initialization:}
The initial population is generated using a predefined value with some added noise to ensure diversity. This is crucial for exploring the solution space.
\[
x_i = x_0 + \mathcal{N}(0, \sigma)
\]
where \( x_i \) is an individual in the initial population, \( x_0 \) is the predefined value, and \( \mathcal{N}(0, \sigma) \) is a random noise value from a normal distribution with mean 0 and standard deviation \( \sigma \).


\subsection{Results}
